\chapter{INTRODUCTION}
Since the seminal report by \citet{Reid1910}, it has been recognized that earthquakes are the sudden release of elastic strain energy that has gradually accumulated over time along tectonic plate boundaries. If we assume that strain is accumulating at a constant rate, then modern-day geodetic observations of tectonic deformation can be used to infer fundamentally important parameters relevant to seismic hazard, such as the recurrence interval for major earthquakes \citep[e.g.,][]{Savage1973,Meade2005}. Indeed, such an assumption has been made in constraining the latest, and most authoritative, seismic hazard model for California, UCERF3 \citep{Field2014}. However, plate tectonics is a more dynamic process, and we can better understand earthquakes, as well as the Earth itself, by considering the time-dependence of tectonic deformation. For example, if we consider a simple of model plate tectonics where an elastic plate is overlying a viscous asthenosphere, then ground deformation should be most rapid immediately after an earthquake while seismic stresses are being relaxed in the asthenosphere \citep{Nur1974,Savage1978}. Observations of this postseismic ground deformation can be used to constrain the strength of the asthenosphere, which can then be used to better understand how stress accumulates on faults. Some of the first observations of postseismic deformation came from triangulation and trilateration surveys \citep[e.g,][]{Thatcher1983}. These geodetic techniques were limited to detecting deformation within a few centimeter precision and at sampling intervals on the order of years. With the advent of space geodetic techniques such as Global Navigation Satellite Systems (GNSS)\footnote{The terms ``GNSS" and ``GPS" (Global Positioning System) will be used synonymously in this dissertation, although GPS is specifically operated by the United States and GNSS is a more generic term.}, geodetic observations have become more precise and have achieved higher sampling rates. Furthermore, in the past decade a dense network of geodetic instruments, consisting of GNSS stations and borehole strain meters, have been installed along the Western United States as part of the Plate Boundary Observatory (PBO). This unprecedented data coverage and data quality has lead to recent discoveries of subtle and short-lived geophysical processes, such as slow slip events along the Cascadia subduction zone \citep{Dragert2001}. Slow slip events occur deep on the subduction zone and they tend to migrate stresses up onto the seismogenic portion of the fault. Slow slip events thus pose a serious risk of triggering megathrust earthquakes, and should be factored in to assessments of seismic hazard. In this dissertation I study transient deformation associated with tectonic processes, such as postseismic deformation and slow slip events. My immediate objective is to detect transient deformation and better understand the mechanisms causing it, while the broader motivation for my research is to improve our understanding of earthquakes and seismic hazard.        

This dissertation can be broadly separated into model-based and data-based analysis. Chapters 2, 3, and 4 are model-based, where I discuss how physical models of tectonic processes can be constrained by observable ground deformation. Chapters 5, 6, and 7 are data-based, where I am mostly focused on assessing the noise in geodetic data and detecting transient geophysical processes, such as slow slip events.    




 there was little reason to believe that plate This data appeared consistent with a constant rate of strain accumulation \citep[e.g.,][]{Savage1973}. With the advent of         

Indeed, the geodetic data available back then, such as triangulation surveys, supported this hypothesis.


The assumption that plate tectonics is constant has lead to fundamental geophysical advancements. Deviations from a steady rate are transients and they provide info.

Transient ground deformation is ...

Transient ground deformation is detected by
Geodesy
 - GNSS (GPS) data
 - BSM
 - note that this is possible from UNVACO and PBO effort
 

The chapters of this dissertation can be broken up into two sections, model based (inverse problems) and data analysis.

In chapter 2 I ....

In chapter 3 I ...

In chapter 4 I ...

In chapter 5 I ...

In chapter 6 I ...

Concluding thoughts


I am a graduate student in geophysics and my dissertation is on detecting and modeling ground deformation in tectonically active regions.  The data in my research is primarily derived from the Global Positioning System (GPS), which can be used to make daily measurements of ground deformation to within millimeter-level accuracy.  With GPS, and other sources of geodetic data, we can observe subtle variations in the rate of ground deformation over time-scales ranging from days to decades.  My research has focused on developing new ways to detect these transient events and, once detected, determine the physical mechanisms driving them.  In doing so, I am improving our understanding of how the earth’s crust behaves mechanically, which then improves our ability to assess seismic hazard.  My dissertation will consist of five chapters.  In my first chapter, I discuss a bias inherent in existing estimates of the mechanical properties of the crust and  upper mantle which have been derived from observable ground deformation.  In my second chapter, I introduce a method for inferring the physical mechanisms driving transient ground deformation following large earthquakes.  In my third chapter, I discuss previously unrecognized ground deformation following the 2010 El Mayor-Cucapah earthquake and I use my method from chapter two to describe its physical cause.  My fourth chapter introduces a Bayesian approach for calculating time dependent crustal strain (i.e. extension and contraction), which is a valuable metric used for detecting transient ground deformation.  My fifth chapter, which I anticipate finishing this coming summer, involves using my derived crustal strain in the Pacific Northwest to describe the fault dynamics of the Cascadia subduction zone.

The first chapter of my dissertation, which was also my first publication, presents a theoretical discussion on what can and cannot be said about the crust and upper mantle based on observable ground deformation.  In particular, this chapter is concerned with the ground deformation that accumulates at tectonic plate margins over decadal time scales.  Several studies have used the spatial and temporal pattern of this ground deformation to infer the strength of the crust and upper mantle. These studies have almost unanimously concluded that the crust is relatively strong compared to the upper mantle, because such a configuration predicts ground deformation that is consistent with the observations.  This chapter urges caution in jumping to the conclusion that the crust is relatively strong.  We demonstrate with numerical modeling that some of the typical assumptions made in these studies could potentially introduce a bias towards inferring the crust to be stronger than it actually is.     

My second and third dissertation chapters focus on transient ground deformation that can be observed following large earthquakes, which is called postseismic deformation.  I analyze five years of postseismic deformation following the 2010 El Mayor-Cucapah earthquake in Baja California.  One of my major findings is that an accelerated rate of deformation can be observed several hundred kilometers away for the earthquake epicenter.  While this far reaching deformation is not immediately recognizable from the raw GPS data, we are able to observe a coherent signal after using a Kalman filtering strategy that we developed and describe in this chapter.  We then analyze this postseismic signal to determine the physical mechanism causing it.  Describing the physical mechanism causing postseismic deformation is a notoriously difficult task because there are often multiple processes occurring simultaneously. Postseismic deformation is typically attributed to continued slow slip on the fault that ruptured during the earthquake, which is termed afterslip.  For large earthquakes, postseismic deformation can be due to an increased rate of viscous flow in the crust and mantle as the earth is relaxing the stress imparted by the earthquake. It is difficult to discern whether afterslip or viscous relaxation is driving postseismic deformation and I devote my second chapter to developing a method for doing so.  My third chapter is an application of this method to the postseismic deformation following the El mayor-Cucapah earthquake.  We find that the far field deformation is best described by viscous flow in the mantle below 50 kilometers depth. In this analysis, we are also able to infer the viscous properties of the crust and mantle in southern California.  The viscous properties of the earth are crucial for reliably assessing seismic hazard because they determine how and where seismic energy builds up in the crust.  Most of our information on viscosity comes from extrapolation of rock experiments performed in laboratories, while the viscosities inferred in this chapter can be viewed as more direct observations. 

My current research, which will become my fourth chapter, presents a method for detecting transient ground deformation from geodetic data.  There are several reasons why one would be interested in detecting transient deformation.  In addition to being a signal that could be used for inferring the mechanical properties of the earth, as described above, several studies have suggested that transient deformation can precede large earthquakes.  While the existence of transient precursors to large earthquakes is debatable, it is sufficient motivation to develop automated techniques for detecting transient deformation.  For these reason, the Southern California Earthquake Center (SCEC) recently hosted the Geodetic Transient-Detection Validation Exercise. In this exercise, simulated geodetic data was used to test some of the state-of-the-art methods for transient detection.  One general approach used by participants was to calculate the time dependent rate of crustal strain, from which transient events can be detected based on a strain threshold criterion.  One problem with this approach is that existing methods for calculating strain lack a meaningful quantification of uncertainty, making it difficult to determine whether an anomalous strain event is statistically significant.  

The method I present for detecting transient deformation is, in essence, a Bayesian approach to calculating time dependent crustal strain.  This method is Bayesian because it requires a prior assumption about the characteristic length-scale and time-scale of the transient features we seek to detect. I then combine the prior assumptions with the geodetic observations to produce a posterior time dependent map of crustal strain.  Most importantly, the derived strain has meaningful and well quantified uncertainties.  One common complaint with Bayesian approaches is that the posterior solution is strongly dependent on the prior assumptions, which can be subjective.  However, since my prior is made explicit and easily interpretable, one could accept or reject the posterior strain map depending on whether they believe that the prior is appropriate.  I demonstrate my method for detecting transient deformation by applying it to geodetic data from the Pacific Northwest.  The fault running beneath this region features periodic earthquakes which occur so slowly that they are only perceptible in geodetic data.  I demonstrate that my method is able illuminate the ground deformation resulting from these slow slip events with unprecedented detail. Furthermore, I am able to draw conclusions about how slow slip events are contributing to the crustal strain in this region. This provides insight into where energy is being built up, which could eventually be released in the next major earthquake.     

The last chapter of my dissertation will be on research that I anticipate doing this summer.  I plan to continue working in the Pacific Northwest and I will use my derived time dependent strain to learn more about the dynamic behavior of the fault running beneath this region.  This fault, which is called the Cascadia subduction zone, is the interface between the North American tectonic plate and the Juan de Fuca plate.  I want to explore how well the two plates are mechanically coupled at the Cascadia subduction zone and I want to explore the time dependence of this coupling.  If we understand how well the plates are coupled then we can better quantify the seismic energy that is being accumulated in this region.  Additionally, the time dependence of the fault coupling can be used to further constrain the frictional properties of the Cascadia subduction zone, which can be a significant factor in assessing seismic hazard yet it is not well understood.     

I have accomplished a lot since I began my doctoral program five years ago. I have learned and developed powerful new techniques which I applied to detecting and modeling transient deformation.  My research has resulted in three publications and a manuscript that will soon be submitted.  The research that I have already accomplished should be sufficient for me to defend my dissertation at the end of this spring semester.  However, with the funding from the Rackham Predoctoral Fellowship I can continue my research over the summer term and finish the projects that I have planned. If I am provided with enough time to finish my current and anticipated work on crustal strain, then it should culminate in two publications that will be well received and impactful in the geophysical community. 
  

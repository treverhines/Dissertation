\chapter{INTRODUCTION}
Since the seminal report by \citet{Reid1910}, it has been recognized that earthquakes are caused by the sudden release of elastic strain energy that has gradually accumulated over time along tectonic plate boundaries. If one assumes that strain is accumulating at a constant rate, then modern-day geodetic observations of tectonic deformation can be used to infer parameters that are fundamentally important for assessing seismic hazard, such as long term fault slip rates \citep[e.g.,][]{Savage1973,Meade2005}. Indeed, such an assumption has been made in constraining the latest, and most authoritative, seismic hazard model for California, UCERF3 \citep{Field2014}. However, plate tectonics is a more dynamic process, and we can better understand earthquakes, as well as the Earth itself, by considering the time-dependence of tectonic deformation. 

Consider a simple model of plate tectonics where an elastic plate is overlying a viscous substrate, representing the ductile lithosphere. Immediately after an earthquake, stresses will be perturbed in the substrate and ground deformation should be most rapid while these coseismic stresses are being viscously relaxed \citep{Nur1974,Savage1978}. Shear stresses will also be increased on the fault surrounding the area that ruptured in the earthquake. These heightened shear stresses can be released seismically through aftershocks or aseismically through afterslip \citep{Marone1991}. Observations of ground deformation resulting from these postseismic processes, termed postseismic deformation, can then be used to constrain the strength of the lithosphere and the frictional properties of faults. Both of which can then be used to better understand how elastic strain energy is accumulating on faults. 

Some of the first observations of postseismic deformation came from leveling and triangulation surveys \citep[e.g,][]{Kanamori1973,Thatcher1975}. These geodetic techniques detected deformation to within an accuracy of a few centimeters and at sampling intervals on the order of years. With the advent of space geodetic techniques, such as Global Navigation Satellite Systems (GNSS)\footnote{The terms ``GNSS" and ``GPS" (Global Positioning System) will be used synonymously in this dissertation, although GPS is specifically operated by the United States and GNSS is a more generic term.}, geodetic observations have become more precise and have achieved higher sampling rates. Furthermore, in the past decade a dense network of geodetic instruments, consisting of GNSS stations and borehole strain meters, have been installed along the Western United States as part of the Plate Boundary Observatory (PBO). This unprecedented data coverage and data quality has improved our ability to resolve postseismic deformation and has lead to the recent discovery of slow slip events along the Cascadia subduction zone \citep{Dragert2001}. Slow slip events occur deep on the subduction zone and they tend to migrate stresses up onto the seismogenic portion of the fault. Slow slip events thus pose a serious risk of triggering megathrust earthquakes and should be factored in to assessments of seismic hazard. 

In this dissertation I study transient deformation associated with tectonic processes, such as postseismic deformation and slow slip events. The immediate objective of this dissertation is to detect transient deformation and better understand the mechanisms causing it. The broader motivation for my research is to improve our understanding of earthquakes and seismic hazard. The chapters of this dissertation can be separated into model-based and data-based analysis. Chapters 2, 3, and 4 are model-based, where I discuss how physical models of tectonic processes can be constrained by observable transient deformation. Chapters 5, 6, and 7 are data-based, where I am mostly focused on assessing the noise in geodetic data and detecting transient geophysical signal.    

Chapter 2 presents a theoretical discussion on what can and cannot be said about the strength of the lithosphere based on interseismic deformation. Interseismic deformation is considered to be deformation over the entire earthquake cycle (100s of years), whereas the postseismic deformation mentioned above tends to refer to deformation in the days to years following an earthquake. Both types of deformation exhibit transients, although interseismic transients is over a much longer timescale. Several studies have used interseismic deformation to infer the strength of the crust and upper mantle. These studies have almost unanimously concluded that the crust is relatively strong compared to the upper mantle \citep{Thatcher2008}. In this chapter I urge caution when interpreting the results of such studies, because I demonstrate that their methodology is inherently biased towards overestimating and underestimated the strength of the crust and upper mantle, respectively. The bias results from the commonly simplifying employed assumption that the viscosities are uniform within the lower crust and upper mantle. 

Chapter 3 and 4 focus on modeling postseismic deformation, where my goal is to infer the lithospheric viscosity and distribution of afterslip from observable postseismic deformation. Such an inverse problem tends to be computationally intractable because there are many unknown parameters being estimated. Additionally, the forward problem is nonlinear with respect to the unknown parameters, and it must be solved with numerical methods. In Chapter 3 we present a approximation for the postseismic deformation forward problem that can be used to make the inverse problem computationally tractable. We demonstrate that our method is able to accurately identify the stength of the lithospheric from postseismic deformation even while afterslip is obscuring the signal from viscous relaxation. Chapter 4 is an application of this method, where I analyze five years of postseismic deformation following the 2010 El Mayor-Cucapah earthquake in Baja California. One of my major findings in Chapter 4 is that an accelerated rate of deformation can be observed several hundred kilometers away for the earthquake epicenter. While this far reaching deformation is not immediately recognizable from the raw GPS data, I am able to observe a coherent signal after using a Kalman filtering strategy that I developed and describe in this chapter. I then analyze this postseismic signal to determine the physical mechanism causing it. I find that the far field deformation is best described by a Burgers rheology upper mantle with a transient viscosity of about $10^{18}$ Pa$\cdot$s. My finding that the mantle is best described with a Burgers rheology, rather than the more commonly employed Maxwell rheology, is more consistent with mantle viscosities determined in other types of geophysical studies \citep[e.g.,][]{Crittenden1967,Bills1987}.    

Chapter 5 is a short, but impactful, note on quantifying noise in geodetic time series.  While this chapter does not directly discuss transient deformation, it is necessary to accurately quantify the noise in geodetic data before any transient signal can be identified. We discuss a bias in a commonly used method for quantifying noise in geodetic data. This bias tends to underestimate the amplitude of noise, which then results in underestimated uncertainties for any geophysical parameters derived from the data. We demonstrate that the Restricted Maximum Likelihood (REML) technique, which was first introduced by \citet{Patterson1971}, can be used as an alternative unbiased method for characterizing noise in geodetic data.

In Chapter 6 we discuss a non-parametric method for detecting transient deformation, specifically transient strain, in GNSS data. Existing transient detection methods tend to assume a parametric form for the underlying signal being detected \citep[e.g.,][]{Ohtani2010}, and an improperly chosen parameterization can bias the results. Our method for detecting transient deformation uses Gaussian process regression, where we assume a stochastic prior model for the underlying signal. As opposed to a subjectively chosen parametric model, our stochastic prior model can be objectively chosen using maximum likelihood techniques, such as the REML method discussed in Chapter 5. We demonstrate our method by applying it to GNSS data from the Pacific Northwest, where we detect transient strain from slow slip events. We verify the accuracy of our detection method by comparing our inferred geophysical signal to observations of seismic tremor, which are known to coincide with slow slip events.       

Chapter 7 is a discussion on borehole strain meters (BSMs) and their ability to record strain from slow slip events. The PBO contains 82 BSMs; however, the data recorded by these instruments is seldom used. This is perhaps because it is unclear whether BSM data, which describes strains over a 8.7 centimeter baseline, are representative of regional tectonic strains. We address this question by comparing GNSS derived strains, described in Chapter 6, to the strains recorded at BSMs. We find that only two BSMs record data that is consistent with the regional strains derived from GNSS data. A third station can be made consistent with the GNSS derived strains if we assume that it is misoriented.

Finally, in Chapter 8 we discuss future research directions and provide concluding remarks.                 
\chapter{INTRODUCTION}
Since the seminal report by \citet{Reid1910}, it has been recognized that earthquakes are the sudden release of elastic strain energy that has gradually accumulated over time along tectonic plate boundaries. If we assume that strain is accumulating at a constant rate, then modern-day geodetic observations of tectonic deformation can be used to infer fundamentally important parameters relevant to seismic hazard, such as the recurrence interval for major earthquakes \citep[e.g.,][]{Savage1973,Meade2005}. Indeed, such an assumption has been made in constraining the latest, and most authoritative, seismic hazard model for California, UCERF3 \citep{Field2014}. However, plate tectonics is a more dynamic process, and we can better understand earthquakes, as well as the Earth itself, by considering the time-dependence of tectonic deformation. For example, if we consider a simple of model plate tectonics where an elastic plate is overlying a viscous asthenosphere, then ground deformation should be most rapid immediately after an earthquake while seismic stresses are being relaxed in the asthenosphere \citep{Nur1974,Savage1978}. Observations of this postseismic ground deformation can be used to constrain the strength of the asthenosphere, which can then be used to better understand how stress accumulates on faults. Some of the first observations of postseismic deformation came from triangulation and trilateration surveys \citep[e.g,][]{Thatcher1983}. These geodetic techniques were limited to detecting deformation within a few centimeter precision and at sampling intervals on the order of years. With the advent of space geodetic techniques such as Global Navigation Satellite Systems (GNSS)\footnote{The terms ``GNSS" and ``GPS" (Global Positioning System) will be used synonymously in this dissertation, although GPS is specifically operated by the United States and GNSS is a more generic term.}, geodetic observations have become more precise and have achieved higher sampling rates. Furthermore, in the past decade a dense network of geodetic instruments, consisting of GNSS stations and borehole strain meters, have been installed along the Western United States as part of the Plate Boundary Observatory (PBO). This unprecedented data coverage and data quality has lead to recent discoveries of subtle and short-lived geophysical processes, such as slow slip events along the Cascadia subduction zone \citep{Dragert2001}. Slow slip events occur deep on the subduction zone and they tend to migrate stresses up onto the seismogenic portion of the fault. Slow slip events thus pose a serious risk of triggering megathrust earthquakes, and should be factored in to assessments of seismic hazard. In this dissertation I study transient deformation associated with tectonic processes, such as postseismic deformation and slow slip events. My immediate objective is to detect transient deformation and better understand the mechanisms causing it, while the broader motivation for my research is to improve our understanding of earthquakes and seismic hazard. 

The chapters of this dissertation can be separated into model-based and data-based analysis. Chapters 2, 3, and 4 are model-based, where I discuss how physical models of tectonic processes can be constrained by observable ground deformation. Chapters 5, 6, and 7 are data-based, where I am mostly focused on assessing the noise in geodetic data and detecting transient geophysical processes, such as slow slip events.    

Chapter 2 presents a theoretical discussion on what can and cannot be said about the strength of the lithosphere based on interseismic deformation. Interseismic deformation is considered to be deformation over the entire earthquake cycle (100s of years), whereas the postseismic deformation mentioned above tends to refer to deformation in the days to years following an earthquake. Both types of deformation exhibit transience, although interseismic transience is over a much longer timescale. Several studies have used interseismic deformation to infer the strength of the crust and upper mantle. These studies have almost unanimously concluded that the crust is relatively strong compared to the upper mantle \citep{Thatcher2008}. In this chapter we urge caution when interpreting the results of such studies, because we demonstrate that their methodology is inherently biased towards overestimating and underestimated the strength of the crust and upper mantle, respectively. The bias results from a simplifying assumption that the lower crust and upper mantle have homogenous, rather than depth dependent, viscosities. A depth dependent viscosity is to be expected since the lithospheric viscosity increases with temperature, and hence depth. 

Chapter 3 and 4 focus on postseismic deformation. While postseismic deformation can be due to viscous relaxation, as described above, it can also be due to slow slip around the portion of the fault the ruptured in the earthquake. It can be difficult to discern which mechanism is driving postseismic deformation. An inverse problem that aims to simultaneously infer the contribution of afterslip and viscous relaxation to postseismic deformation can be numerically expensive. In Chapter 3 we present a approximation for postseismic deformation that can be used to make such an inverse problem computationally tractable. We demonstrate that the viscosity of the lithosphere can accurately estimated from postseismic deformation even while afterslip is obscuring the signal from viscous relaxation. Chapter 4 is an application of this method, where I analyze five years of postseismic deformation following the 2010 El Mayor-Cucapah earthquake in Baja California. One of my major findings in chapter 4 is that an accelerated rate of deformation can be observed several hundred kilometers away for the earthquake epicenter. While this far reaching deformation is not immediately recognizable from the raw GPS data, we are able to observe a coherent signal after using a Kalman filtering strategy that we developed and describe in this chapter. We then analyze this postseismic signal to determine the physical mechanism causing it. We find that the far field deformation is best described by a Burgers rheology upper mantle with a transient viscosity of about $10^{18}$ Pa$\cdot$s. Our finding that the mantle is best described with a Burgers rheology, rather than the more commonly employed Maxwell rheology, is consistent with mantle viscosities determined in other types of geophysical studies, such as studies on isostatic adjustment of the lithosphere after Lake Bonneville was drained \citep{Crittenden1967,Bills1987}.    

Chapter 5 is a short, but impactful, note on quantifying noise in geodetic time series.  While this chapter does not directly discuss transient deformation, it is necessary to accurately quantify the noise in geodetic data before any transient signal can be identified, which we discuss in Chapter 6. We discuss a bias that is inherent in the commonly used method for quantifying noise in geodetic data. This bias tends to underestimate the amplitude of noise, which then results in underestimated uncertainties in any geophysical parameters derived from the data. We demonstrate that the Restricted Maximum Likelihood (REML) technique, which was first introduced by \citep{Patterson1971}, can be used as an alternative unbiased method for characterizing noise.

In Chapter 6 we discuss a non-parametric method for detecting transient deformation, specifically transient strain, in GNSS data. Existing transient detection methods tend to assume a parametric form for the underlying signal being detected \citep[e.g.,][]{Ohtani2010}, and an improperly chosen parameterization can bias the results. Our method for detecting transient deformation uses Gaussian process regression, where we assume a stochastic prior model for the underlying signal. As opposed to a subjectively chosen parametric model, our stochastic prior model can be objectively chosen using maximum likelihood techniques, such as the REML method discussed in Chapter 5. We demonstrate our method using GNSS data in the Pacific Northwest, where we detect transient strain from slow slip events. We verify the accuracy of our detection method by comparing our inferred geophysical signal to observations of seismic tremor, which are known to coincide with slow slip events.       

Chapter 7 is a discussion on borehole strain meters (BSMs) and their ability to record strain from slow slip events. The PBO contains 82 BSMs; however, the data recorded by these instruments is seldom used. The underutilization of BSMs is perhaps because it is unclear whether BSM data, which describes strains over a 8.7 centimeter baseline, are representative of regional tectonic strains. We address this question by comparing GNSS derived strains, which are described in Chapter 6, to the strain recorded at BSMs. We find that only two BSMs record data that is consistent with the regional strains derived from GNSS data. A third station can be made consistent with the GNSS derived strains if we assume that it is mis-oriented.

Finally in chapter 8 we discuss future research directions and provide concluding remarks.                 
\chapternonum{ABSTRACT}
Modern-day geodetic tools, such as global navigation satellite systems
(GNSS), are able to detect tectonic ground deformation to within
millimeter accuracy. In the past decade, our ability to resolve ground
deformation in the Western United States has greatly improved with the
Plate Boundary Observatory (PBO) project, which consists of about 1100
continuously operating GNSS stations. With this unprecedented quality
and quantity of data, we can observe the subtle signal from transient
tectonic processes. For example, we can detect ground deformation in
the days to years following large earthquakes, which is caused by
aseismic afterslip and/or ductile deformation in the upper mantle. The
PBO has also allowed us to resolve transient deformation associated
with slow slip events on the Cascadia subduction zone. In this
dissertation, I discuss detecting transient deformation in geodetic
data, and I use this transient deformation to better understand the
mechanical behavior of the crust and upper mantle.

Studies on the temporal evolution of ground deformation throughout the
earthquake cycle (i.e., interseismic deformation) tend to infer that
the lower crust is orders of magnitude stronger (more viscous) than
the upper mantle. In Chapter 2, I demonstrate that the methods used in
these studies are biased towards inferring a more viscous lower crust
and less viscous lower mantle. I conclude that these interseismic
studies do not necessarily rule out the possibility that the lower
crust can deform ductilely on earthquake cycle timescales. In Chapters
3 and 4, I introduce a method for discerning the physical mechanisms
driving postseismic deformation, where I consider candidate mechanisms
to be afterslip and viscous relaxation in the crust and upper mantle.
I apply this method to postseismic deformation following the 2010 El
Mayor-Cucapah earthquake in Baja California. I find that a Burgers
rheology upper mantle is necessary to describe far-field deformation
in the three years following the earthquake. In general, lithospheric
viscosities inferred from interseismic deformation are larger than
those estimated from postseismic deformation, which occurs over a much
shorter timescale. By describing the upper mantle with a Burgers
rheology, which contains a transient and steady-state phase of
deformation, I am able to reconcile these conflicting studies.
Chapters 5, 6, and 7 are on detecting transient geophysical signal in
geodetic data. Before geophysical signal can be identified, the noise
in geodetic data must be quantified. In Chapter 5, I point out a bias
in the commonly used method for characterizing noise in geodetic data,
and I propose an alternative unbiased method. In Chapter 6, I
demonstrate that transient geophysical signal can be robustly detected
by using a machine learning technique known as Gaussian process
regression. Finally, in Chapter 7 I assess the utility of borehole
strain meters (BSMs) for detecting transient deformation. I find that
BSM data, which records strains over an 8.7 centimeter baseline, are
difficult to reconcile with regional strains derived from GNSS data.

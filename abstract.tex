\chapternonum{ABSTRACT}
Transient deformation, resulting from either aseismic fault slip or
distributed ductile deformation, offers valuable insight into the
mechanical behavior of the lithosphere. In this dissertation we
provided guidance on detecting and interpreting transient deformation.
In Chapter 2 we explore the accuracy of existing constraints on the
strength of the lithosphere determined from interseismic deformation.
We have concluded that the method typically used to infer lithospheric
strength from interseismic deformation is biased towards finding a
more viscous lower crust and a less viscous upper mantle. In Chapter 3
introduce a method for discerning the mechanisms driving postseismic
deformation. Chapter 4, is an application of the method from Chapter 3
to postseismic deformation following the 2010 El Mayor-Cucapah
earthquake in Baja California. We found that a Burgers rheology upper
mantle is necessary to describe far-field postseismic deformation.
This finding helps to reconcile a discrepancy between the lithospheric
rheology inferred from postseismic studies and interseismic studies.
In general, studies of interseismic deformation find that the mantle
is significantly stronger than estimates made from postseismic
studies. Our inferred Burger's rheology reconciles this discrepancy
because we can explain interseimic deformation with the steady-state
viscosity, and postseismic deformation, which occurs over shorter
timescales, can be explained with the transient viscosity.

Chapters 5, 6, and 7 are concerning detecting geophysical signal in
geodetic data. Before geophysical signal can be identified, the noise
in geodetic data must be quantified. In Chapter 5, I point out a bias
in the maximum likelihood estimation (MLE) method typically used for
quantifying noise in geodetic data. I then demonstrate the efficacy of
the restricted maximum likelihood (REML) method as an unbiased
alternative. In Chapter 6, I show that Gaussian process regression is
a powerful tool for detecting transient geophysical signal in GNSS
data. This method is able to successfully identify strain from slow
slip events, and it provides an unprecedented view of how strain
evolves during slow slip events. Our inferences of transient strain
reveal features in the GNSS dataset that could potentially be used to
constrain our understanding of how faults behave. In Chapter 7 we have
analyzed data from borehole strain meters (BSMs) and we found that it
is generally difficult to reconcile BSM data with the regional strain
derived from GNSS data. Of the several dozen BSMs in the Pacific
Northwest, only two stations record strain that is consistent with
GNSS data.
